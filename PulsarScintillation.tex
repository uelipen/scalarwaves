
\documentclass[prl,aps,floatfix,superscriptaddress,twocolumn]{revtex4}

\usepackage{amssymb,amsmath}
\usepackage{graphicx}
\usepackage[percent]{overpic}
\usepackage{overpic}

\begin{document}  



\title{ Testing Gravity with Pulsar Scintillation Measurements} 



\author{Huan Yang,$^{1,2}$ Atsushi Nishizawa,$^{3} $Ue-Li Pen$^{4}$\\
%}
%
%\affiliation{
$^{1}$Perimeter Institute for Theoretical Physics, Waterloo, ON N2L2Y5, Canada\\
$^{2}$Institute for Quantum Computing, University of Waterloo, Waterloo,
ON N2L3G1, Canada\\
$^{3}$Califoria Institute of Technology, Pasadena, CA 91125, United States \\
$^{4}$ Canadian Institute for Theoretical Astrophysics, 60 St. George Street,
Toronto, On M5S3H8, Canada
}


\date{\today}





\begin{abstract}
We propose to use pulsar scintillation measurements to test predictions of alternative theories of gravity. Comparing to single-path pulsar timing measurements, the scintillation measurements can achieve a factor of $10^4 \sim 10^5 $ improvement in timing accuracy, due to the effect of multi-path interference. The self-noise from pulsar also does not affect the interference pattern, where the data acquisition timescale is $10^3$ seconds instead of years. Therefore it has unique advantages in measuring gravitational effect or other mechanisms (at mHz and above frequencies) on light propagation. We illustrate its application in constraining scalar gravitational-wave background and measuring gravitational-wave speed, in which cases the sensitivities are greatly improved with respect to previous limits. We expect much broader applications in testing gravity with existing and future pulsar scintillation observations. 
\end{abstract}

\maketitle 

\noindent{\bf Introduction} -- Pulsar scintillation happens when pulsed radio signals from pulsars follow different paths of propagation to reach earth, and it exists for almost all known pulsars. It is generally believed that, turbulent interstellar plasma residing on the propagation path plays the role of an effective ``lens", generates necessary diffraction and scattering for pulses along different paths to meet at earth. Upon arrival, these radio signals interfere with each other and generate a spatially varying interference pattern.  As earth is moving, an telecope observer experiences time-dependent intensitiy variation corresponding to different fringes in the interference pattern.

Follow the illustration plot Fig.~\ref{fig:pulsarpath}, the spatial separation between fringes is approximately $\lambda/\alpha$ ($\lambda$ is the radio wavelength) and the temporal seperation is $\sim \lambda/(\alpha V_e)$, where $V_e$ is the projected earth velocity. Therefore at GHz band and with path openning agnle $\alpha$ assumed to be $\sim {\rm arcsec}$, one typically observes hundreds or more fringes within the timescale of a single pulse. By statisticaly (see the discussion in the next section) averaging over time shift of the fringes, it is possible to achieve pico-second resolution in time, which is a factor of $10^5$ higher than the accuracy in single-path pulsar timing.


%such interference pattern contains many fringes within the timescale of a single pulse, which can be statistically averaged out to achieve pico-second order timing resolution. 
This unpresendetned timing precession allows the applications of scintillations in probing the physics of turbulent plasma in the interstellar medium \cite{Rickett1977, Rickett1986}  and constraining the size of emission regions in the pulsar magnetospheres \cite{Johnson2012}. Although high precision pulsar timing has been discussed extensively in literature to test alternative theories of gravity, little was known in relating scintillation measurements to test gravity.  In this Letter, we propose to use pulsar scintilattion measurements as a laborotory for gravitaional physics, in particular, testing the effects generated by gravitatinal waves (GWs). Similar analysis can be applied to test other physical  effects that affect the propagation of radio waves.

\vspace{0.2cm}

\noindent {\bf Scintillation Noise} -- {\bf Ue-Li: Please rewrite this section, including a brief discussion on different noise in scintillation measurements and how to obtain the frequency dependent timing noise. After that we can give number for some particular pulsar as an example.} The scintillation of pulsars contain several noise components. The first noise piece comes from the emission source, which is usually referred as the {\it self noise} []. Such noise is generated by the stochastic processes in the emission sources, and its contamination to the visibility coulld be reduced by applying statisical averaging []. Secondly, there are additional complications due to intrinsic variations in the intensity of pulses, the finite size effect of the emission domain and the decoherence due to turbulent plasma motions.

The scintillation pattern has a typical decoehrence bandwidth $\Delta \nu_{\rm d} \propto \lambda^{-4}$ and  timescale $\Delta t_{\rm d} \sim s_0/V \propto \lambda^{-1}$, where $s_0$ is the spatial correlation length, $V$ is the earth velocity and $\lambda$ is the wavelength of the radio wave. The observational bandwidth $B$ is assumed to be much larger than $\Delta \nu_{\rm d}$, and the data is averaged within a group of $N$ adjacent pulses. If we want to probe physics at frequency $f$, it is necassary to have $N t_0 \le f^{-1} \le \Delta t_{\rm d}$, where $t_0$ is the period for pulsar duty cycles.   Therefore at higher frequencies the number of available pulses in each averaging group is smaller, and the total number of pulses (for a typical milli-second pulsar) within the scintillation timescale is of order $10^6$. After the statisitical averaging [], the signal-to-noise ratio ${\rm SN}$ improves by a factor of $\sqrt{N}$.

The frequency-dependent timing accuracy is related to the bandwidth and the signal to noise ratio. {\bf Please add the reasoning}. Based on the above reasoning we have the following simplified model for the timing noise
\begin{align}\label{eqt}
\delta t_{\rm f} =  \left\{
\begin{array}{c}
1 ps \times \left ( \frac{\delta t_{\rm mHz}}{1 ps} \right ) \times \left (\frac{f}{ 10^{-3 } Hz} \right )^{1/2}\,,   \qquad f\ge1/\Delta t_{\rm d}\, \\
\\
\infty\,, \qquad \qquad f < 1/\Delta t_d \,.
\end{array}\right.
\end{align}
where $\delta t_{\rm mHz}$ is the timing accuracy at $mHz$. In realitity there is still some residual sensitivity for $f<1/\Delta t_{\rm d}$, but we ignore this regime for simplicity. For Vela Pulsar $\delta t_{\rm mHz}$ is about. 




\begin{figure}[t]
  \begin{overpic}[width=0.9\columnwidth]{illustration.pdf}
\end{overpic}
  \caption{(Color Online). The illustration for pulsed signals that arrive on earth following two distinctive paths, where the wave following $\mathcal{C}'$ is deflected by the interstellar medium at location ``D". Here $L_1 = r L$ and $L_2 = (1-r)L$. When the radio waves reach earth observer from these two directions, they interfere and produce very fine interference pattern based on the radio wavelength $\lambda$ and the path openning angle $r \alpha $. As earth moves at a speed $V_e \sim 30 km/s$, there are many fringes within the timescale of a single pulse (for illustration purpose we only show a few finges within each pulse).  }
	\label{fig:pulsarpath}
\end{figure}


\noindent {\bf Probing nontensorial components of GWs} -- According to the theory of General Relatvity, gravitational waves only have two tensor polarizations that are transverse to the wave propagation direction. However, gravitaional wave emissions with scalar and vector polarizations are predicted in many alternative theories of gravity, such as Scalar-Tensor Theory, $f(R)$ theory, bimetric theory, etc. A summary about GW polarization prediction in various alternative gravity models can be found in \cite{Nishizawa2009} and reference therein. 

Therefore measuring and/or constraining GWs with non-tensor polarizations is a viable approach to test the theories of gravity and serach for possible new physics. At ten to several hundred Hz, i.e., LIGO frequency band, it has been suggested \cite{Nishizawa2009,Nishizawa2013} to use a joint measurement of serval ground-based GW detectors to separate the mixture of different polarization modes. Between $10^{-4}$Hz and $1$ Hz the current best constraint on nontensorial gravitational wave background (GWB) is given by acceleration measurements of satellites, with constraint of dimensionless GW amplitude (both tensorial and nontensorial mode) on the order of $10^{-15}$. At even lower frequency (nano Hz and below) the only only available constraints come from Pulsar Timing Array (PTA) measurements and anisotropy measurements of Cosmic Microwave Background. As an illustration for applications of pulsar scintillation observations in testing gravity, we show that the existing data already provides the best constraint on scalar GWB at mHz band, which beats the previous constraint by a factor of $10^4$ but might be improved by future space-based GW missions such as eLISA. In order to obtain the sensitivity curve, we will show how to derive the transfer functions in such systems, while similar analysis may be applicable for other examples of using pulsar scintillations to test gravity.

As shown in Fig. \ref{fig:pulsarpath}, we consider a train of radio waves emitted from Pulsar ("P") propagates along two different patha ($\mathcal{C}$ and $\mathcal{C}'$) and eventually reaches earth. For simplicity, we only consider an one-time deflection by the turbulent plasma at location "D", and we assume both paths are on the $x-z$ plane, with $\mathcal{C}$ being along $x$ axis. The coordinate of "P","D",and "O" on the $x-z$ plane is $[0,0],[L r/(1+r), L r \alpha/(1+r)], [L, 0]$ respectively.

Based on the standard pulsar timing analysis, the GW-induced (frequency $\omega$)  phase shift of radio waves propagating along $\mathcal{C}$  is (hereafter we adopt the geometric unit that the speed of light $c=1$)
\begin{align}\label{eqc}
H_C = \frac{\pi n^i h_{ij} n^j}{\omega \lambda} \frac{\sin [\omega L \xi+\psi]-\sin \psi}{\xi}\,, 
\end{align}
where $\psi$ is the initial phase of that particular GW, $\xi \equiv 1-{\bf k}\cdot {\bf n}$, with ${\bf k}$ being the direction vector of the GW and ${\bf n}={\bf e}_x$ being the direction vector of $P\rightarrow O$. The metric perturbation $h_{\mu\nu}$ has $10$ components, $4$ of which are purely gauge. We can exploit the gauge freedom to let $h_{0\mu}=0$ and there are $6$ degrees of freedom left in $h_{ij}$ ($1 \le i,j \le 3$). We follow the convention in \cite{Nishizawa2009, Lee:2013} to label these $6$ polarizations ($2$ tensor modes: $+,\times$, $2$ vector modes: $x,y$ and $2$ scalar modes: $b, l$). In the case that GW is propagating along $z$-axis, the tensor basis  is
\begin{align}
&\tilde{e}^+
= 
\left (
\begin{array}{c c c}
1 & 0 & 0 \\
0 & -1 & 0 \\
0 & 0 & 0
\end{array}
\right ),\quad
\tilde{e}^\times
=\left (
\begin{array}{c c c}
0 & 1 & 0 \\
1 & 0 & 0 \\
0 & 0 & 0
\end{array}
\right )\,, \nonumber \\
&\tilde{e}^b
= 
\left (
\begin{array}{c c c}
1 & 0 & 0 \\
0 & 1 & 0 \\
0 & 0 & 0
\end{array}
\right ),\quad
\tilde{e}^l
=\left (
\begin{array}{c c c}
0 & 0 & 0 \\
0 & 0 & 0 \\
0 & 0 & 1
\end{array}
\right )\,, \nonumber \\
&\tilde{e}^x
= 
\left (
\begin{array}{c c c}
0 & 0 & 1 \\
0 & 0 & 0 \\
1 & 0 & 0
\end{array}
\right ),\quad
\tilde{e}^y
=\left (
\begin{array}{c c c}
0 & 0 & 0 \\
0 & 0 & 1 \\
0 & 1 & 0
\end{array}
\right )\,, 
\end{align}
so that $h_{ij}$ can be decomposed as 
\begin{align}
h_{ij} = \sum_{m} h_m \tilde{e}^m_{ij}\,.
\end{align}  
Follow the same princple, the phase shift (due to the same GW train) of radio waves propagating along $\mathcal{C}'$ is
\begin{align}\label{eqcp}
H_{\mathcal{C}'}= &\frac{\pi n_1^i h_{ij} n_1^j}{\omega \lambda} \frac{\sin [\omega r L \xi_1+\psi]-\sin \psi}{\xi_1}\nonumber\\
&+\frac{\pi n_2^i h_{ij} n_2^j}{\omega \lambda} \frac{\sin [\omega L \xi+\psi]-\sin [\omega r L \xi_1+\psi]}{\xi_2}\,,
\end{align}
where ${\bf n}_1={\bf e}_x+\alpha {\bf e}_z$, ${\bf n}_2 ={\bf e}_x-r \alpha  {\bf e}_z$ and $\xi_{1,2}=1-{\bf n}_{1,2} \cdot {\bf k}$. With $H_{\mathcal{C}}$ and $H_{\mathcal{C}'}$, we can derive the phase shift after averaging over sky direction and initial phase of the GW  
\begin{align}
\delta \Phi^2 &\equiv \langle (H_{\mathcal{C}}-H_{\mathcal{C}'})^2\rangle \nonumber \\
&\propto  \frac{h^2_m \alpha^2}{\omega^2 \lambda^2} \left\{
\begin{array}{cl}
 \log(\omega L)\,, & m=+, \times\, b, \\
\\
 \omega L\,, & m =x, y\,, \\
\\
\omega^2L^2\,, & m=l\,,
\end{array}\right.
\label{bx}
\end{align}
assuming $\omega L \gg 1$ (at mHz band it's greater than $10^{10}$ for typical pulsars). In partular, we find that the longitudinal mode (``$l$") receives the largest amplification factor ($\propto \omega^2 L^2$), as all other polarizations generate transverse displacement with respect to the GW propagation direction. The transfer function for longitudinal scalar mode is given by
\begin{align}
\delta \Phi =\frac{ \pi h_l \alpha L}{\lambda}\sqrt{\frac{r}{2(1+r)}} \sqrt{ \log (1+r)+r\log \frac{1+r}{r}}.
\end{align}
Combining this transfer function  with the timing noise estimate given in the previous section, we can obtain the sensitivity of pulsar scintillation measurement on longitudinal scalar GWs, by making $\delta \Phi =2 \pi c\, \delta t_{\rm f}/\lambda$. Take $r \sim 1$, this gives the sensitivity on $h_l$ (dimensionless GW amplitude) as
\begin{align}
h_l = 6.8\times 10^{-18}  \frac{\delta t_{\rm mHz}}{1 ps}   \left (\frac{f}{ m Hz} \right )^{1/2} \left (\frac{\alpha}{{\rm arcsec}} \right )^{-1} \left ( \frac{L}{k {\rm pc}}\right)^{-1}\,.
\end{align}

\begin{figure}[t]
  \begin{overpic}[width=0.9\columnwidth]{sensitivity.pdf}
\end{overpic}
  \caption{(Color Online). The constraint of dimensionless amplitude of longitudinal scalar GWs. Blue dotted, Black solid and Red dashed lines correspond to sensitivity curves given by previous Doppler tracking of sattelites, pulsar scintillation discussed in this work and future LISA measurements respectively.  }
	\label{fig:sensitivity}
\end{figure}


In Fig.~\ref{fig:sensitivity}, we compare the sensitivity of longitudinal scalar GWs based on scnitillation measurements of Vela Pulsar {\bf We will update the plot with new timing noise formula and a better pulsar at larger distance} with the current best constraint based on satellite/spacecraft timing and the proposed sensitivity of eLISA at the same frequency band. For satellite timing and eLISA, We have presented the sensitivity of  tensor modes in the figure, as the transfer functions of scalar longitudinal modes and tensor modes are approximately the same below $0.1$ Hz, as shown in \cite{Tinto2010}. In addition, we have combined the estimation from Dopper tracking of the {\it Cassini} spacecraft in \cite{Bertotti1995,Armstrong2003} amd timing measurement of the GPS system in \cite{Aoyama2014} to obtain previous constraint. We can see that scinetillation measurement from vela pulsar already improves the previous sensitivity by a factor of $10^3 \sim 10^4$.  By choosing more distant pulsars and/or the ones with better scintillation timing accuracy, as well as stastitically averaging data for different scintillating pulsars,  it is possible to further improve this limit.
\vspace{0.2cm}

\noindent{\bf Other applications} -- As a second application, we discuss using pulsar scintillations to measure possible diffrence between GW speed (both $+,\times$ modes) and light speed. Similar to non-tensorial GW polarizations, it is possible for GWs to propagate at a different speed from light  in many modified gravity models (such as Scalar Tensor Theory). We will avoid discussing different possible physical origins of this deviation, but instead derive its frequency dependent constraint based on scintillation data.

The basic idea is similar to previous work on measuring GW speed, which was based on  Pulsar Timing data and led to constraint at $n$Hz band \cite{Baskaran2008}. We make the assumption that there is a  GWB (of tensorial waves). At the $mHz$ band, the dominant sources for GWB is extragalactic white dwalf and neutron star binaries {\bf Astushi please update this}, which contributes gavitational-wave strain 
\begin{align}\label{eqhc}
h_c(f) \sim 10^{-18} \left ( \frac{f}{m Hz}\right )^{-2/3}\,.
\end{align}

If GW speed is less than the light speed $\epsilon \equiv 1-v_{\rm gw}/c > 0$, the part of GWs with ${\bf n}\cdot {\bf k} \approx 1-\epsilon$ will have similar projected speed of the radio waves, and consequenctly their propagation phases are approximately synronized. The polarization of these GWs are not completely orthogonal to the direction of radio waves, so that these GWs are still able to shift the phase of the radio waves. For single-path Pulsar timing observation, this resonance effect (also known as ``surfing effect" \cite{Aoyama2014, Polnarev2008}) leads to the square phase shift (averaging over the sky directions)
\begin{align}
\delta \Phi^2_s \sim \frac{2\pi^3 \epsilon^2 L h_c^2}{ \omega \lambda^2}\,,
\end{align} 
assuming $\epsilon^2 \omega L \gg 1$. For double-path scintillation scenario, we notice that the phase shift along $\mathcal{C}, \mathcal{C}'$ can be obtained by replacing $\xi = 1-\epsilon-{\bf n}\cdot {\bf k}$ in Eq.~\ref{eqc} and Eq.~\ref{eqcp}. After applying the sky and initial phase average, we can obtain the following transfer function  
\begin{align}\label{eqphid}
\delta \Phi^2_d \sim \frac{\pi^3 r^2}{(1+r)^2} \frac{h_c^2\alpha^2 \epsilon^3 (\omega L)^3}{\omega^2 \lambda^2}\,,
\end{align}
in the limit of $\epsilon \omega L \gg 1$. As we are comparing phase between two colse paths $\mathcal{C}, \mathcal{C}'$, the surfing effect is enhanced iwith the presence of additional $\omega L$. At the same frequency, $\delta \Phi_d/\delta \Phi_s \sim \alpha \omega L\sqrt{\epsilon} \gg 1$ for typical pulsars at $m$Hz band and with $\epsilon \sim 3.7 \times 10^{-3}$ bound given in \cite{Aoyama2014} (obtained at $n$Hz range). Taking into account the fact of better timing accuracy with pulsar scintillations measurements, it is certainly always better to measure $\epsilon(f)$ at $m$Hz or higher frequencies. In fact, combining Eq.~\ref{eqphid}, Eq.~\ref{eqhc} and Eq.~\ref{eqt}, we arrive at the following upper limit on $\epsilon$ (take $r \sim 1$): 
\begin{align}
\epsilon(f) \le& 1.8\times 10^{-4} \nonumber\\ 
&\times \left ( \frac{\delta t_{\rm mHz}}{1 ps}\right )  \left (\frac{f}{ m Hz} \right )^{2/3} \left (\frac{\alpha}{{\rm arcsec}} \right )^{-1} \left ( \frac{L}{k {\rm pc}}\right)^{-3/2}
\end{align}
which gives a first constraint on GW speed at $10^{-3}$Hz - $1$Hz frequency band. We note that for certain models that give rise to nonzero $\epsilon$, e.g. the ones with constant nonzero graviton mass $m_g$, it is much more convenient to use low-frequency tests, as $\epsilon(\omega) \sim m^2_g/\omega^2$. However, in order to obtain model-indepdent constraints, we need to measure $\epsilon(f)$ at different frequencies.
\vspace{0.2cm}

\noindent{\bf Discussion}--  Comparing to single path pulsar timing measurements, the scintillation measurements have better timing accuracies, and the phase-comparison geometry which naturally removes intrinsic noise from the source. These are the key factors which ensures its ultra presicion and enables its application in studying ISM physics, pulsar physics and our proposal in this Letter - testing alternative gravity models.

We have illustrated two examples in this proposal: measuring longitudinal scalar GW and the speed of GWs. It is also possible to apply for other tests which do not involve GWs - for example, 
the holographic noise [] or the spacetime quantum fluctuations [] would contribute distinctive phase noise for photon traveling along different scintillation paths, and hence can be measured by observing analomous scintillation phase shift or degrading of the interference pattern.

{\it Acknowledgements-} We thank I-Sheng Yang for very instructive discussions on timing noise of Pulsar scintillations and Nestor Ortiz for making Fig.1. HY acknowledges supports from the Perimeter Institute of Theoretical Physics and the Institute for Quantum Computing. Research at Perimeter Institute is supported by the government of Canada and by the Province of Ontario though Ministry of Research and Innovation.
\bibliography{References}

%\bibliographystyle{References}



\end{document}




















